% Options for packages loaded elsewhere
\PassOptionsToPackage{unicode}{hyperref}
\PassOptionsToPackage{hyphens}{url}
%
\documentclass[
]{article}
\usepackage{amsmath,amssymb}
\usepackage{lmodern}
\usepackage{iftex}
\ifPDFTeX
  \usepackage[T1]{fontenc}
  \usepackage[utf8]{inputenc}
  \usepackage{textcomp} % provide euro and other symbols
\else % if luatex or xetex
  \usepackage{unicode-math}
  \defaultfontfeatures{Scale=MatchLowercase}
  \defaultfontfeatures[\rmfamily]{Ligatures=TeX,Scale=1}
\fi
% Use upquote if available, for straight quotes in verbatim environments
\IfFileExists{upquote.sty}{\usepackage{upquote}}{}
\IfFileExists{microtype.sty}{% use microtype if available
  \usepackage[]{microtype}
  \UseMicrotypeSet[protrusion]{basicmath} % disable protrusion for tt fonts
}{}
\makeatletter
\@ifundefined{KOMAClassName}{% if non-KOMA class
  \IfFileExists{parskip.sty}{%
    \usepackage{parskip}
  }{% else
    \setlength{\parindent}{0pt}
    \setlength{\parskip}{6pt plus 2pt minus 1pt}}
}{% if KOMA class
  \KOMAoptions{parskip=half}}
\makeatother
\usepackage{xcolor}
\IfFileExists{xurl.sty}{\usepackage{xurl}}{} % add URL line breaks if available
\IfFileExists{bookmark.sty}{\usepackage{bookmark}}{\usepackage{hyperref}}
\hypersetup{
  pdftitle={Affan's WebJournal},
  pdfauthor={Muhammad Affan Qamar},
  hidelinks,
  pdfcreator={LaTeX via pandoc}}
\urlstyle{same} % disable monospaced font for URLs
\usepackage[margin=1in]{geometry}
\usepackage{graphicx}
\makeatletter
\def\maxwidth{\ifdim\Gin@nat@width>\linewidth\linewidth\else\Gin@nat@width\fi}
\def\maxheight{\ifdim\Gin@nat@height>\textheight\textheight\else\Gin@nat@height\fi}
\makeatother
% Scale images if necessary, so that they will not overflow the page
% margins by default, and it is still possible to overwrite the defaults
% using explicit options in \includegraphics[width, height, ...]{}
\setkeys{Gin}{width=\maxwidth,height=\maxheight,keepaspectratio}
% Set default figure placement to htbp
\makeatletter
\def\fps@figure{htbp}
\makeatother
\setlength{\emergencystretch}{3em} % prevent overfull lines
\providecommand{\tightlist}{%
  \setlength{\itemsep}{0pt}\setlength{\parskip}{0pt}}
\setcounter{secnumdepth}{-\maxdimen} % remove section numbering
\ifLuaTeX
  \usepackage{selnolig}  % disable illegal ligatures
\fi

\title{Affan's WebJournal}
\author{Muhammad Affan Qamar}
\date{30-04-2022}

\begin{document}
\maketitle

\textbf{IMPORTANT:} You can delete everything in here and start fresh.
You might want to start by not deleting anything above this line until
you know what that stuff is doing.

This is an \texttt{.Rmd} file. It is plain text with special features.
Any time you write just like this, it will be compiled to normal text in
the website. If you put a \# in front of your text, it will create a top
level-header.

\hypertarget{my-second-post-note-the-order}{%
\section{My second post (note the
order)}\label{my-second-post-note-the-order}}

Last compiled: 2022-04-30

\hypertarget{adding-r-stuff}{%
\section{Adding R stuff}\label{adding-r-stuff}}

So far this is just a blog where you can write in plain text and serve
your writing to a webpage. One of the main purposes of this lab journal
is to record your progress learning R. The reason I am asking you to use
this process is because you can both make a website, and a lab journal,
and learn R all in R-studio. This makes everything really convenient and
in the same place.

So, let's say you are learning how to make a histogram in R. For
example, maybe you want to sample 100 numbers from a normal distribution
with mean = 0, and standard deviation = 1, and then you want to plot a
histogram. You can do this right here by using an r code block, like
this:

When you knit this R Markdown document, you will see that the histogram
is printed to the page, along with the R code. This document can be set
up to hide the R code in the webpage, just delete the comment (hashtag)
from the cold folding option in the yaml header up top. For purposes of
letting yourself see the code, and me see the code, best to keep it the
way that it is. You'll learn that all of these things and more can be
customized in each R code block.

\hypertarget{tidying-data-with-pivot-longer}{%
\section{Tidying Data With pivot
longer}\label{tidying-data-with-pivot-longer}}

diamonds2 \textless- readRDS(``C:/Users/Muhammad Affan
Qamar/Downloads/diamonds2.rds'') diamonds2

\hypertarget{tidying}{%
\subsection{Tidying}\label{tidying}}

diamonds2 \%\textgreater\% pivot\_longer(cols = c(``2008'', ``2009''),
names\_to = `year', values\_to = `price') \%\textgreater\% head(n = 5)

\hypertarget{tidying-data-with-pivot-wider}{%
\section{Tidying Data With pivot
wider}\label{tidying-data-with-pivot-wider}}

diamonds3 \textless- readRDS(``C:/Users/Muhammad Affan
Qamar/Downloads/diamonds3.rds'') diamonds3 \#\# Tidying diamonds3
\%\textgreater\% pivot\_wider(names\_from = ``dimension'' , values\_from
= ``measurement'') \%\textgreater\% head(n=5)

\hypertarget{tidying-data-with-pivot-wider-separate}{%
\section{Tidying Data With pivot wider separate
()}\label{tidying-data-with-pivot-wider-separate}}

diamonds4 \textless- readRDS(``C:/Users/Muhammad Affan
Qamar/Downloads/diamonds4.rds'') diamonds4 \#\#Tidying diamonds4
\%\textgreater\% separate(col = dim, into = c(``x'', ``y'', ``z''), sep
= ``/'', convert = T)

\hypertarget{tidying-data-with-pivot-wider-unite}{%
\section{Tidying Data With pivot wider unite
()}\label{tidying-data-with-pivot-wider-unite}}

diamonds5 \textless- readRDS(``C:/Users/Muhammad Affan
Qamar/Downloads/diamonds5.rds'') diamonds5 \#\# Tidying diamonds5
\%\textgreater\% unite(clarity, clarity\_prefix, clarity\_suffix, sep =
'\,')

\hypertarget{transform}{%
\section{Transform}\label{transform}}

\hypertarget{filter-slice}{%
\subsection{filter(), slice()}\label{filter-slice}}

diamonds \%\textgreater\% filter(cut == `Ideal' \textbar{} cut ==
`Premium', carat \textgreater= 0.23) \%\textgreater\% head(5) diamonds
diamonds \%\textgreater\% filter(cut == `Ideal' \textbar{} cut ==
`Premium', carat \textgreater= 0.23) \%\textgreater\% slice(3:4)
diamonds

\hypertarget{arrange}{%
\subsection{arrange()}\label{arrange}}

diamonds \%\textgreater\% arrange(cut, carat, desc(price))

\hypertarget{select}{%
\subsection{select()}\label{select}}

diamonds \%\textgreater\% select(color, clarity, x:z) \%\textgreater\%
head(n = 5)

\hypertarget{exclusive-select}{%
\subsubsection{Exclusive select}\label{exclusive-select}}

diamonds \%\textgreater\% select(-(x:z)) \%\textgreater\% head(n = 5)

\hypertarget{select-helpers}{%
\subsubsection{Select helpers}\label{select-helpers}}

\hypertarget{starts_with-ends_with}{%
\subsection{starts\_with() / ends\_with()}\label{starts_with-ends_with}}

\texttt{starts\_with()\ /\ ends\_with()} : helper that selects every
column tht starts with a prefix or ends with a suffix

\hypertarget{contains}{%
\subsection{contains()}\label{contains}}

\texttt{contains()} : A select helper that selects any column containing
a string of text.

\hypertarget{everything}{%
\subsection{everything()}\label{everything}}

\texttt{everything()} a select() helper that selects every column that
has not already been selected. Good for reordering.

´´´\{r\} select(x:z, everything()) \%\textgreater\% head(n = 5) ´´´ \#\#
rename() \texttt{rename()}: changes the name of a column.`

\hypertarget{mutate}{%
\subsection{mutate()}\label{mutate}}

\texttt{mutate()} adds new variables that are functions of existing
variables and preserves existing ones.

\hypertarget{transmute}{%
\subsection{transmute()}\label{transmute}}

\texttt{transmute()} adds new variables and drops existing ones.

\end{document}
